\documentclass[hidelinks, a4paper, 12pt]{article}
\usepackage[linktoc=all]{hyperref}
\usepackage{apacite}
\usepackage[margin=1.0in]{geometry}
\usepackage{amssymb}
\usepackage{amsmath}

% \makeatletter
% \renewcommand{\@dotsep}{10000} 
% \makeatother

\hypersetup{
    pdftitle={Pure Mathematics III: Differential Equations},
    pdfauthor={Wilson Wongso},
    pdfpagemode=UseOutlines,
}

\title{Pure Mathematics III: Differential Equations}
\author{Based on lectures by Danilo J. Alcordo \\ Notes taken by Wilson Wongso}
\date{Junior College 2 - Academic Year 2018/2019}

\setcounter{section}{-1}
\setcounter{tocdepth}{2}

\begin{document}

    \maketitle
    
    \tableofcontents

    \section{Preface}
        The following lecture notes are mostly based on textbook \cite{neill2016cambridge} questions and past-paper questions from various years.
        The author assumes that the readers are able to do different methods of integration, such as indefinite integral, 
        u-substitution/integration by substitution, integration by parts, and partial fraction decomposition.\\[\baselineskip]
        These notes only include the key parts of the lectures and the types of problems that often appear in the actual exam.
        Further reading and past-year papers practice are highly encouraged.

    \section{Definitions}
        \subsection{Differential Equations in Science}
        A differential equation is a \textbf{mathematical equation} that relates some \textbf{function}
        with its \textbf{derivatives}. In applications, the functions usually represent physical quantities,
        the derivatives represent their rates of change, and the equation defines a relationship
        between the two. Because such relations are extremely common, differential equations play
        a prominent role in many disciplines including \textbf{engineering}, \textbf{physics}, 
        \textbf{economics}, and \textbf{biology}.

        \subsection{Differential Equations in Pure Mathematics}
        In \textbf{pure mathematics}, differential equations are studied from several different perspectives,
        mostly concerned with their solutions - the set of functions that satisfy the equation. Only the simplest
        differential equations are solvable by explicit formulas; however, some properties of solutions of
        a given differential equation may be determined without finding their exact form.
    
    \section{General Solution of a Differential Equation}
        \subsection{Introduction}
            A general solution of an nth-order equation is a solution containing n arbitrary independent constants of integration \cite{kreyszig1999advanced}.
        \subsection{Methods of solving a differential equation}
            \subsubsection{Separation of independent and dependent variables}
                A derivative can be separated into its independent and dependent variables.
                Separating them allows us to turn the differential equation into a form which we know how to solve.
                For instance,
                \[\frac{dy}{dx} = f(y) \Leftrightarrow \frac{dx}{dy} = \frac{1}{f(y)} \Leftrightarrow dx = \frac{1}{f(y)}dy \Leftrightarrow x = \int \frac{1}{f(y)}dy\]
        \subsection{Example problems}
            Find the general solution of the following differential equation.
            \subsubsection{Problem 1}
                \[\frac{dy}{dx} = (3x - 1)(x-3)\]
                \textbf{Solve by Separating Variables}. Expand the equation.
                \[\frac{dy}{dx} = 3x^2 - 10x + 3\]
                Separate the variables with respect to $y$ and with respect to $x$.
                \[dy = (3x^2 - 10x + 3)dx\]
                Integrate both sides.
                \[\int dy = \int (3x^2 - 10x + 3)dx\]
                Thus, we obtain the general solution:
                \[y = x^3 - 5x^2 + 3x + k\]
            \subsubsection{Problem 2}
                \[\frac{dy}{dt} = sin^2(3t)\]\\
                Separate the variables with respect to $y$ and with respect to $t$.
                \[dy = sin^2(3t)dt\]
                Integrate both sides.
                \[\int dy = \int sin^2(3t)dt\]
                Substitute $u = 3t$.
                \[Let: u = 3t\]
                \[du = 3dt\]
                \[dt = \frac{1}{3} du\]
                Obtaining:
                \[\int dy = \frac{1}{3}\int sin^2(u)du\]\\
                Use the trigonometric identity $2sin^2(\theta) \equiv 1 - cos(2\theta)$ to get:
                \[\int dy = \frac{1}{3}\int \frac{1}{2}(1-cos(2u))du\]
                \[\int dy = \frac{1}{6}\int (1-cos(2u))du\]
                \[\int dy = \frac{1}{6}\int du -\frac{1}{6}\int cos(2u)du\]
                Substitute $v = 2u$.
                \[Let: v = 2u\]
                \[dv = 2du\]
                \[du = \frac{1}{2} dv\]
                Obtaining:
                \[\int dy = \frac{1}{6}\int du -\frac{1}{12}\int cos(v)dv\]
                \[y = \frac{1}{6}u - \frac{1}{12}sin(v) + k\]
                Substitute back the variables $u$ and $v$ into $t$ accordingly:
                \[y = \frac{1}{6}3t - \frac{1}{12}sin(2*3t) + k\]
                Finally obtaining:
                \[y = \frac{1}{2}t - \frac{1}{12}sin(6t) + k\]
                as our general solution.
    \section{Particular Solution of a Differential Equation}    
        \subsection{Introduction}
            A particular solution is derived from the general solution by setting the constants to particular values, often chosen to fulfill set 'initial conditions or boundary conditions' \cite{kreyszig1999advanced}.
        \subsection{Example Problems}
            Solve the following differential equations with the given initial conditions.
            \subsubsection{Problem 1}
                \[\frac{dx}{dt} = 2e^{0.4t},\]
                $x = 1$ when $t = 0$\\[\baselineskip]
                \textbf{Solution}
                \[\frac{dx}{dt} = 2e^{0.4t}\]
                Separate the variables.
                \[dx = 2e^{0.4t}dt\]
                Integrate.
                \[\int dx = 2\int e^{0.4t}dt\]
                \[x = \frac{2}{0.4}e^{0.4t} + k\]
                Obtaining: 
                \[x = 5e^{0.4t} + k\]
                Then, we know $x = 1$ when $t = 0$, so we plug those values into the general solution:
                \[1 = 5e^{0} + k\]
                \[1 = 5 + k\]
                To get the actual value of $k$:
                \[k = -4\]
                Replace $k$ in the general solution with the previous value, to get:
                \[x = 5e^{0.4t} - 4\]
                as our particular solution.
            \subsubsection{Problem 2}
                \[(1-t^2)\frac{dy}{dt} = 2t,\]
                $y = 0$ when $t = 0$, for $-1 < t < 1$\\[\baselineskip]
                \textbf{Solution}
                \[(1-t^2)\frac{dy}{dt} = 2t\]
                Separate the variables.
                \[dy = \frac{2t}{1-t^2}dt\]
                \[\int dy = \int \frac{2t}{1-t^2}dt\]
                There are multiple approaches to this integral, the longer method being \textbf{partial fraction decomposition}:\\[\baselineskip]
                \textbf{Method 1: Partial Fraction Decomposition}
                    \[\frac{2t}{1-t^2} \equiv \frac{2t}{(1+t)(1-t)}\]
                    which we can transform to the form:
                    \begin{equation}
                        \frac{2t}{(1+t)(1-t)} = \frac{A}{1+t} + \frac{B}{1-t}
                    \end{equation}
                    To get the values of $A$ and $B$, we first need to select one of the denominators and set it equal to zero:
                    \begin{equation}
                        1+t = 0   
                    \end{equation}
                    \begin{equation}
                        \Rightarrow t = -1
                    \end{equation}
                    Then, multiply both sides of equation $(1)$ by the selected denominator:
                    \[\frac{2t}{1-t} = A + \frac{B(1+t)}{1-t}\]
                    Since we set $1+t = 0$ from equation $(2)$, we get:
                    \begin{equation}
                        \frac{2t}{1-t} = A
                    \end{equation}
                    Lastly, we use the value of $t$ from equation $(3)$ into $(4)$:
                    \[\frac{2(-1)}{1-(-1)} = A\]
                    And finally:
                    \[A = -1\]
                    Repeat the same process with a different denominator to obtain the value of $B$, such that:
                    \[B = 1\]
                    With the values of $A$ and $B$, we can finally return to our integration, which is now of the form:
                    \[\int dy = \int \frac{1}{1-t} - \frac{1}{1+t}dt\]
                    \[\int dy = \int \frac{1}{1-t}dt - \int \frac{1}{1+t}dt\]
                    Then integrate,
                    \[y = -ln|1-t| - ln|1+t| + k\]
                    \[y = -(ln|1-t| + ln|1+t|) + k\]
                    Obtaining:
                    \[y = -(ln|1-t^2|) + k\]
                    as our general solution.\\[2\baselineskip]
                \textbf{Method 2: U-Substitution}
                    \[\int dy = \int \frac{2t}{1-t^2}dt\]
                    \[Let: u = 1-t^2\]
                    \[du = -2tdt\]
                    \[-du = 2tdt\]
                    \textbf{Notice}: $2tdt$ is the numerator of our fraction in the right hand side, hence:
                    \[\int dy = -\int \frac{1}{u}du\]
                    \[y = -ln|u| + k\]
                    As a result, we get:
                    \[y = -ln|1-t^2| + k\]
                    as our general solution.\\[2\baselineskip]
                    \textbf{Note:} Both approaches works, but not everyone might notice the u-substitution that
                    can make the integral much simpler, thus the author would like to show both methods.\\[2\baselineskip] 
                \textbf{Finding the particular solution:}
                    \[y = -ln|1-t^2| + k\]
                    We know $y = 0$ when $t = 0$,
                    \[0 = -ln|1-0| + k\]
                    \[0 = -ln(1) + k\]
                    \[0 = 0 + k\]
                    \[k = 0\]
                    Lastly, we just need to replace $k$ with its value, to find:
                    \[y = -ln|1-t^2|\]
                    as our particular solution.
    \section{Differential Equations as Essay Problems}
        Normally, some differential equations come in the form of an essay or story-like problem. So, in addition
        to solving the equation, we need to firstly, come up with the correct differential equation, according to
        what the essay tells us.
        \subsection{Example Problems}
            \subsubsection{Problem 1}
                In starting from rest, the driver of an electric car depresses the throttle gradually. If the speed
                of the car after $t$ seconds is $v$ $ms^{-1}$, the acceleration $\frac{dv}{dt}$ (in metre-second units)
                is given by $0.2t$. How long does it take for the car to reach a speed of $20$ $ms^{-1}$?\\[\baselineskip]
                \textbf{Solution}\\
                    In order to answer the last section of the problem, we need to find out the relationship between quantities $v$
                    and $t$, or a function $v$ dependent on $t$.
                    \[v(t)\]
                    To find that, we need to solve a differential equation. In this case, the problem explicitly gives us the differential equation it requires:
                    \[\frac{dv}{dt} = 0.2t\]
                    However, there is actually an implicit initial condition that the problem also gives.
                    The problems states "starting from rest," which we can imply that in \textbf{rest} the car does not
                    move and that moment is the starting time of when we take the quantity $t$. Hence:
                    \[v = 0\] when \[t = 0\]
                    Now, we can finally solve the differential equation and find its particular solution!
                    \[\frac{dv}{dt} = 0.2t\]
                    \[dv = 0.2tdt\]
                    Integrate: 
                    \[\int dv = 0.2\int tdt\]
                    \[v = \frac{0.2}{2}t^2 + k\]
                    General Solution:
                    \[v = 0.1t^2 + k\]
                    Use the initial conditions $v = 0$ and $t = 0$:
                    \[0 = 0 + k\]
                    \[k = 0\]
                    Hence, the particular solution:
                    \begin{equation}
                        v = 0.1t^2
                    \end{equation}
                    Some students stop here, but what the question also requires us to answer is the time it takes
                    for the car to reach a speed of $20$ $ms^{-1}$. We can transform equation $(5)$ where the dependent variable is now $v$:
                    \[t = \sqrt{10v}\]
                    Now to obtain the last mark, we need to find $t$ at $v=20$ $ms^{-1}$:
                    \[t = \sqrt{10*20}\]
                    \[t = \sqrt{200}\]
                    And finally get:
                    \[t \approx 14.1s\]
                    as our final answer.
            \subsubsection{Problem 2}
                The solution curve for a differential equation of the form $\frac{dy}{dx} = x - \frac{a}{x^2}$ for $x > 0$,
                passes through the points $(1, 0)$ and $(2, 0)$. Find the value of $y$ when $x = 3$.\\[\baselineskip]
                \textbf{Solution}\\
                Like \textbf{4.1.1 Problem 1}, the problem gives the differential equation explicitly. However, it now involves a constant $a$
                and later $k$, the constant of integration.
                \[\frac{dy}{dx} = x - \frac{a}{x^2}\]
                Separate the variables:
                \[dy = x - \frac{a}{x^2}dx\]
                Integrate:
                \[\int dy = \int x - \frac{a}{x^2}dx\]
                \[\int dy = \int xdx -a \int \frac{1}{x^2}dx\]
                General Solution:
                \begin{equation}
                    y = \frac{1}{2}x^2 + \frac{a}{x} + k   
                \end{equation}
                Notice how the number of initial conditions given is the same as the number of constants in our general solution. This calls
                for a system of linear equation of two variables, which we can solve using methods such as substitution or elimination. The two 
                given points are:
                \[(1, 0)\] and \[(2, 0)\]
                At point $(1, 0)$ equation $(6)$ yields:
                \begin{equation}
                    0 = \frac{1}{2} + a + k
                \end{equation}
                At point $(2, 0)$ equation $(6)$ yields:
                \begin{equation}
                    0 = 2 + \frac{a}{2} + k
                \end{equation}
                From equation $(8)$, we can multiply both sides by $2$ and get the expression:
                \[0 = 4 + a + 2k\]
                \begin{equation}
                    a = -4 -2k
                \end{equation}
                Using $(9)$, we can substitute $a$ in $(7)$ to find the value of $k$:
                \[0 = \frac{1}{2} - 4 - k\]
                \[k = -\frac{7}{2}\]
                and substituting the value of $k$ into $(9)$ yields:
                \[a = 3\]
                With both values of the constants obtained, we can finally find the particular solution:
                \[y = \frac{1}{2}x^2 + \frac{3}{x} - \frac{7}{2}\]
                However, the problem also requires us to find the value of $y$ when $x = 3$,
                \[y = \frac{1}{2}3^2 + \frac{3}{3} - \frac{7}{2}\]
                Thus as the final answer:
                \[y = 2\]
            \subsubsection{Problem 3}
                A girl lives 500 metres from school. She sets out walking at 2$ms^{-1}$, but when she has
                walked a distance of $x$ metres, her speed has dropped to $(2 - \frac{1}{400}x) ms^{-1}$.
                How long does she take to get to school?\\[\baselineskip]
                \textbf{Solution}\\
                Unlike the two previous problems, now we don't explicitly get the differential equation.
                But, we can indeed infer from the problem. The quantity \textbf{speed}, as we know from Physics,
                is just the change in \textbf{distance} over \textbf{time}. Thus, we can say that speed is equivalent
                to $\frac{dx}{dt}$ or sometimes denoted by $\dot{x}$ in Physics. Therefore initially:
                \[\frac{dx}{dt} = 2\]
                But as the problem tells us, it dropped to:
                \[\frac{dx}{dt} = 2 - \frac{1}{400}x\]
                We can now solve the differential equation. Firstly, make the expression on the right hand side into one fraction:
                \[\frac{dx}{dt} = \frac{800-x}{400}\]
                Separate the variables:
                \[\frac{1}{800-x}dx = \frac{1}{400}dt\]
                Integrate:
                \[\int \frac{1}{800-x}dx = \frac{1}{400}\int dt\]
                General Solution:
                \[-ln|800-x| = \frac{1}{400}t + k\]
                Now you might be wondering on how we can find the value of $k$, but the problem implies two things; 
                that before the girl started walking (at rest) $t = 0$, and distance $x$ is of course $0$,
                because she hasn't started walking towards school. With that, we get:
                \[x = 0\] when \[t = 0\]
                Thus:
                \[-ln|800| = 0 + k\]
                \[k = -ln(800)\]
                Substituting the value of $k$ into the general solution, yields:
                \[-ln|800-x| = \frac{1}{400}t -ln(800)\]
                as our particular solution.
                Yet as usual, the problem doesn't end there. It is asking for the duration it takes for the girl to reach school.
                From the first sentence of the problem, we know that the distance $x$ the girl will travel to school is 500m. Hence
                at $x = 500$:
                \[-ln(800-500) = \frac{1}{400}t -ln(800)\]
                \[-ln(300) = \frac{1}{400}t -ln(800)\]
                \[ln(800)-ln(300) = \frac{1}{400}t\]
                \[ln\left(\frac{800}{300}\right) = \frac{1}{400}t\]
                \[t = 400ln\left(\frac{8}{3}\right)\]
                And finally obtain:
                \[t \approx 392.33s \approx 6.5min\]
    \section{Past Paper Questions}
        \subsection{Problem 1}
            A biologist is investigating the spread of a weed in a particular region. At time $t$ weeks after the start of
            the investigation, the area covered by the weed is $A$ $m^2$. The biologist claims that the rate of increase of $A$
            is proportional to $\sqrt{(2A-5)}$.\\[\baselineskip]
            \textbf{i)} Write down a differential equation representing the biologist's claim. \textbf{[1]}\\[\baselineskip]
            \textbf{ii)} At the start of the investigation, the area covered by the weed was $7$ $m^2$ and, 10 weeks later, the
            area covered was $27$ $m^2$. Assuming that the biologist's claim is correct, find the area covered 20 weeks after the
            start of the investigation. \textbf{[9]}\\[\baselineskip]
            Before we solve, we need to know more about proportionality. For example, in Physics, Hooke's Law tells us that the force 
            needed to extend or compress a spring by some distance $x$ scales linearly or proportionally with respect to that distance.
            This relation is represented by:    
            \[F_{s} = kx\]
            where $F_{s}:$ spring force; $k:$ spring constant; $x:$ spring stretch or compression\\[\baselineskip]   
            \textbf{Note: }If a rate of change is proportional to a certain quantity, then there exists a constant that can represent
            the proportionality between the two quantities.\\[\baselineskip]
            \textbf{Note: }If the rate is increasing proportionally, the constant is positive. While if the rate is decreasing proportionally,
            the constant is negative.\\[\baselineskip]
            \textbf{Solution}\\[\baselineskip]
            \textbf{i)} $\frac{dA}{dt} = k\sqrt{(2A-5)}$\\[\baselineskip]
            \textbf{ii)} Separate the variables:
            \[\frac{1}{\sqrt{2A-5}}dA = kdt\]
            Integrate both sides:
            \begin{equation}
                \int \frac{1}{\sqrt{2A-5}}dA = k\int dt   
            \end{equation}
            Use the substitution $u = 2A-5$:
            \[Let: u = 2A - 5\]
            \[du = 2dA\]
            \[dA = \frac{1}{2}du\]
            Hence equation $(10)$ turns into:
            \[\frac{1}{2}\int \frac{1}{\sqrt{u}}du = k\int dt\]
            \[\frac{1}{2}\int u^{-1/2}du = k\int dt\]
            \[u^{1/2} = kt + c\]
            \[\sqrt{u} = kt + c\]
            Substitute $u$ back into $2A-5$ yields the general solution:
            \begin{equation}
                \sqrt{2A-5} = kt + c
            \end{equation}
            Like previously, we need to find 2 sets of initial conditions because of the 2 unknown constants present in our general solution.
            We know from section \textbf{(ii)}, that at the start, the area covered by the weed was 7$m^2$. Thus:
            \[A = 7\] when \[t = 0\]
            With that, we can substitute them into equation $(11)$ to obtain the value of $c$:
            \[\sqrt{14-5} = c\]
            \[\sqrt{9} = c\]
            \[c = 3\]
            Yielding the equation:
            \begin{equation}
                \sqrt{2A-5} = kt + 3
            \end{equation}
            Lastly to find $k$, section \textbf{(ii)} also tells us that at $t = 10$, $A = 27$. Substituting them into equation $(12)$ gives:
            \[\sqrt{54-5} = 10k + 3\]
            \[\sqrt{49} = 10k + 3\]
            \[4 = 10k\]
            \[k = \frac{2}{5}\]
            Finally, we have:
            \[\sqrt{2A-5} = \frac{2}{5}t + 3\]
            \[2A-5 = {\left(\frac{2}{5}t + 3\right)}^2\]
            \[2A = {\left(\frac{2}{5}t + 3\right)}^2 + 5\]
            \[A = \frac{1}{2}{\left(\frac{2}{5}t + 3\right)}^2 + \frac{5}{2}\]
            as our particular solution.\\[\baselineskip]
            To completely answer the problem, we need to find the value of $A$ at $t = 20$:
            \[A = \frac{1}{2}{\left(\frac{2}{5}20 + 3\right)}^2 + \frac{5}{2}\]
            \[A = \frac{121}{2}+ \frac{5}{2}\]
            \[A = \frac{126}{2}\]
            \[A = 63\]
            Hence we get $A = 63$ as our final answer for section \textbf{(ii)}.
        \subsection{Problem 2}
            The number of organisms in a population of time $t$ is denoted by $x$. Treating $x$ as
            a continuous variable, the differential equation satisfied by $x$ and $t$ is $\frac{dx}{dt}=\frac{xe^{-t}}{k+e^{-t}}$, where $k$ is a
            positive constant.\\[\baselineskip]
            \textbf{i)} Given that $x = 10$ when $t = 0$, solve the differential equation, obtaining a relation between $x$, $k$ and $t$. \textbf{[6]}\\[\baselineskip]
            \textbf{ii)} Given also that $x = 20$ when $t = 1$, show that $k = 1 - \frac{2}{e}$. \textbf{[2]}\\[\baselineskip]
            \textbf{iii)} Show that the number of organisms never reaches 48, however large $t$ becomes. \textbf{[2]}\\[\baselineskip]
            \textbf{Solution}\\[\baselineskip]
            \textbf{i)} Given the differential equation:
            \[\frac{dx}{dt}=\frac{xe^{-t}}{k+e^{-t}}\]
            Separate the variables:
            \[\frac{1}{x}dx=\frac{e^{-t}}{k+e^{-t}}dt\]
            Integrate both sides:
            \begin{equation}
                \int \frac{1}{x}dx=\int \frac{e^{-t}}{k+e^{-t}}dt
            \end{equation}
            Use the substitution $u = k + e^{-t}$:
            \[Let: u = k + e^{-t}\]
            \[du = -e^{-t}\]
            \[-du = e^{-t}\]
            \textbf{Notice}: Numerator $e^{-t}dt = -du$.\\
            Apply the substitution into $(13)$:
            \[\int \frac{1}{x}dx=-\int \frac{1}{u}du\]
            \[ln|x| = -ln|u| + c\]
            Substitute $u$ back into $k + e^{-t}$ to get the general solution:
            \[ln|x| = -ln|k + e^{-t}| + c\]
            From section \textbf{(i)}, we know that $x = 10$ when $t = 0$, so after substitution:
            \[ln(10) = -ln|k+e^0| + c\]
            \[ln(10) = -ln|k+1| + c\]
            \[c = ln(10) + ln|k+1|\]
            \[c = ln|10k + 10|\]
            Hence, we get:
            \[ln|x| = -ln|k + e^{-t}| + ln|10k + 10|\]
            \[ln|x| = ln|\frac{10k+10}{k + e^{-t}}|\]
            \begin{equation}
                x = \frac{10k+10}{k + e^{-t}}
            \end{equation}
            As the relation between $x$, $k$, and $t$.\\[\baselineskip]
            \textbf{ii)} With the relation we just obtained, we can substitute the initial conditions $x = 20$ when $t = 1$ into equation $(14)$:
            \[x = \frac{10k+10}{k + e^{-t}}\]
            \[20 = \frac{10k+10}{k + e^{-1}}\]
            \[20(k + e^{-1}) = 10k+10\]
            \[20k + 20e^{-1} = 10k+10\]
            \[10k = 10 - 20e^{-1}\]
            \[k = 1 - 2e^{-1}\]
            \[k = 1 - \frac{2}{e}\]
            which is exactly as what the question required us to show.\\[\baselineskip]
            \textbf{iii)} We can prove that $x$ will never reach 48, by a few methods:\\[\baselineskip]
            \textbf{Method 1: Using a large value for $t$}\\[\baselineskip]
            Firstly, we need to evaluate the value of $k$:
            \[k = 1 - \frac{2}{e}\]
            \[k \approx0.264\]
            Then, set $t = 1000$ (or any arbitrarily large value):
            \[t = 1000\] 
            Lastly, substitute those values into equation $(14)$:
            \[x = \frac{10(0.264)+10}{(0.264) + e^{-1000}}\]
            Obtaining:
            \[x \approx 47.8\]
            Thus:
            \[x < 48\]
            \textbf{Method 2: Evaluate $\lim_{t\to\infty}$ of equation (14)}\\[\baselineskip]
            Similar to \textbf{Method 1}, we need to evaluate the value of $k$ and substitute into $(14)$:
            \[k \approx0.264\]
            \[x = \frac{10(0.264)+10}{(0.264) + e^{-t}}\]
            Then find the limit as $t$ approaches $\infty$:
            \[\lim_{t\to\infty} \left(x = \frac{10(0.264)+10}{0.264 + e^{-t}}\right)\]
            \[x = \lim_{t\to\infty}\left(\frac{10(0.264)+10}{0.264 + e^{-t}}\right)\]
            The only quantity that relies on $t$ is $e^{-t}$, hence the limit is only required to be evaluated there:
            \[x = \frac{10(0.264)+10}{0.264 + \lim_{t\to\infty}e^{-t}}\]
            As we know:
            \[\lim_{t\to\infty}e^{-t}\]
            \[\Rightarrow \lim_{t\to\infty}\frac{1}{e^t}\]
            \[=0\]
            Therefore:
            \[x = \frac{10(0.264)+10}{0.264 + 0}\]
            \[x \approx 47.8\]
            And:
            \[x < 48,\]
            which implies that $x$ will never reach 48 however large $t$ becomes.
    \bibliographystyle{apacite}
    \bibliography{References}
    
\end{document}