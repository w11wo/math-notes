\documentclass[hidelinks, a4paper, 12pt]{article}
\usepackage[linktoc=all]{hyperref}
\usepackage{apacite}
\usepackage[margin=1.0in]{geometry}
\usepackage{amssymb}
\usepackage{amsmath}
\usepackage{amsthm}
\newtheorem{theorem}{Theorem}
\usepackage{pgfplots}
\pgfplotsset{compat=1.16}
\usepackage{tikz}
\usepackage{pgf}
\usepackage{mathrsfs}
\usepackage{array}
\usepackage{tabularx}
\usepackage{braket}

\allowdisplaybreaks


\hypersetup{
    pdftitle={Wavefunctions: Position and Momentum},
    pdfauthor={Wilson Wongso},
    pdfpagemode=UseOutlines,
}

\title{Wavefunctions: Position and Momentum}
\author{Notes taken by Wilson Wongso}
\date{}

\setcounter{section}{-1}
\setcounter{tocdepth}{2}

\graphicspath{ {./images/} }

\newcommand{\biimp}{\Leftrightarrow}
\newcommand{\bd}{\textbf}
\newcommand{\n}{\\[\baselineskip]}
\newcommand{\real}{\mathbb{R}}
\newcommand{\thus}{\Rightarrow}
\newcommand{\xhat}{\hat{x}}
\newcommand{\pxhat}{\hat{p}_x}

\begin{document}

    \maketitle
        
    \tableofcontents

    \section{Preface}
        The following notes are based on the lecture video \bd{Position and Momentum from Wavefunctions} \cite{QM-Position-Momentum}.
        The author simply wishes to compile a part of his learning journey into this document.

    \section{Position}
        Suppose a particle is described by its wavefunction $\psi(x,t)$.\n
        \scalebox{0.95}{
        \begin{tikzpicture}
            \begin{axis}[
                axis lines=middle,
                xlabel=$x$, xlabel style={at=(current axis.right of origin), anchor=west},
                ylabel=$\psi$, ylabel style={at=(current axis.above origin), anchor=south},
                xtick=\empty, ytick=\empty,
                clip=false,
            ]
            \addplot [
                domain=0.5:3.1, 
                samples=100, 
                color=red,
            ]
            {sin(deg(x))};
            \addplot +[mark=none, color=white, smooth] coordinates {(1.2, 0) (1.2, 0.93)};
            \addplot +[mark=none, color=white, smooth] coordinates {(1.9, 0) (1.9, 0.9463)};
            \node[label={0:{$\psi(x,t)$}},circle,inner sep=0pt] at (axis cs:2.4,0.7) {};
            \end{axis}
        \end{tikzpicture}}

        \subsection{Probability of Finding the Position of a Particle}
            \begin{tikzpicture}
                \begin{axis}[
                    axis lines=middle,
                    xlabel=$x$, xlabel style={at=(current axis.right of origin), anchor=west},
                    ylabel=$\psi$, ylabel style={at=(current axis.above origin), anchor=south},
                    xtick=\empty, ytick=\empty,
                    clip=false,
                ]
                \addplot [
                    domain=0.5:3.1, 
                    samples=100, 
                    color=red,
                ]
                {sin(deg(x))};
                \addplot +[mark=none, color=blue, smooth] coordinates {(1.2, 0) (1.2, 0.93)};
                \addplot +[mark=none, color=blue, smooth] coordinates {(1.9, 0) (1.9, 0.9463)};
                \node[label={150:{$a$}},circle,fill,inner sep=2pt] at (axis cs:1.2, 0) {};
                \node[label={20:{$b$}},circle,fill,inner sep=2pt] at (axis cs:1.9, 0) {};
                \end{axis}
            \end{tikzpicture}\n
            The probability that we will find the particle between two points $a$ and $b$, is given by the integral:
            \[\int_a^b |\psi(x,t)|^2dx\]

        \subsection{Expectation Value of the Position of a Particle}
            While the expectation value of the position is given by the integral:
            \begin{equation}
                \braket{x} = \int_{-\infty}^{\infty} x|\psi(x,t)|^2dx
            \end{equation}
            What the expectation value represents is \bd{not} the mean value of position from taking several consecutive measurements of the particle.\n
            This is because consecutive measurements would cause \bd{wavefunction collapse}, turning the wavefunction into a delta-function-like graph.\n
            \begin{tikzpicture}
                \begin{axis}[
                    axis lines=middle,
                    xlabel=$x$, xlabel style={at=(current axis.right of origin), anchor=west},
                    ylabel=$\psi$, ylabel style={at=(current axis.above origin), anchor=south},
                    xtick=\empty, ytick=\empty,
                    clip=false,
                ]
                \addplot [
                    domain=0.5:3.1, 
                    samples=100, 
                    color=white,
                ]
                {sin(deg(x))};
                \addplot [
                    domain=1.58:1.6, 
                    samples=100, 
                    color=red,
                ]
                {50*x-79};
                \addplot [
                    domain=1.6:1.62, 
                    samples=100, 
                    color=red,
                ]
                {-50*x+81};
                \node[label={-90:{$x_1$}},circle,fill,inner sep=2pt] at (axis cs:1.6, 0) {};
                \end{axis}
            \end{tikzpicture}\n
            Instead, the expectation value represents the mean value of position from single measurements of an infinite collection of \bd{identical} particles/wavefunctions/systems, at an exact same time.\n
            Alternatively, we can measure one wavefunction, let the system \bd{recover from the collapse}, take the measurement again, repeat, and then finally take the mean of the position.\n
            Both of these methods allow us to calculate the expectation value as:
            \[\braket{x} = \frac{x_1 + x_2 + .. + x_n}{n}\]
            where $n$ is the number of measurements we took.

        \subsection{Position Operator}
            Now we'd like to find the corresponding position operator $\xhat$, which would be a Hermitian Operator due to the $2^{nd}$ postulate of Quantum Mechanics.\n
            Recall that the expectation value of the position $x$ can be related to its operator $\xhat$ and is defined by the following:
            \begin{equation}
                \braket{x} = \frac{\braket{\psi | \xhat | \psi}}{\braket{\psi | \psi}}
            \end{equation}
            We can then equate $(1)$ to $(2)$ since they both represent the expectation value of $x$:
            \[\frac{\braket{\psi | \xhat | \psi}}{\braket{\psi | \psi}} = \int_{-\infty}^{\infty} x|\psi(x,t)|^2dx\]
            Since the wavefunction is normalized, its inner product with itself is $1$:
            \[\braket{\psi | \xhat | \psi} = \int_{-\infty}^{\infty} x|\psi(x,t)|^2dx\]
            We also know that the magnitude squared of the wavefunction $|\psi|^2$ is just $\psi$ multiplied by its conjugate $\psi^*$. Substituting that and rearranging the right-hand side yields:
            \[\braket{\psi | \xhat | \psi} = \int_{-\infty}^{\infty} \psi^* (x) \psi dx\]
            From here, we can deduce that the operator $\xhat$ is simply given by its position $x$:
            \[\xhat = x\]

    \section{Momentum}
        Similarly, the momentum of a particle can be represented by a Hermitian Operator $\pxhat$.\n
        We can arrive at this operator by firstly finding the expectation value of its momentum, $\braket{p_x}$.

        \subsection{Expectation Value of the Momentum of a Particle}
            Surely enough, we can actually derive $\braket{p_x}$ using the expectation value of its position $\braket{x}$.\n
            We begin by differentiating $\braket{x}$ with respect to time:
            \[\begin{split}
                \frac{d\braket{x}}{dt} &= \frac{d}{dt} \int_{-\infty}^{\infty} x|\psi(x,t)|^2dx\\
                \\
                &= \int_{-\infty}^{\infty} \frac{\partial}{\partial t} [x|\psi(x,t)|^2]dx\\
                \\
                &= \int_{-\infty}^{\infty} x \frac{\partial}{\partial t} [|\psi(x,t)|^2]dx\\
                \\
                &= \int_{-\infty}^{\infty} x \frac{\partial}{\partial t} (\psi^* \cdot \psi)dx\\
            \end{split}\]
            Recall that in the derivation of Schrödinger's Equation, we had already found an expression for $\frac{\partial}{\partial t}(\psi^* \cdot \psi)$, plugging that in yields:
            \[\begin{split}
                \frac{d\braket{x}}{dt} &= \int_{-\infty}^{\infty} x \left[\psi^* \left(\frac{i\hbar}{2m} \frac{\partial^2 \psi}{\partial x^2}\right) - \psi \left(\frac{i\hbar}{2m} \frac{\partial^2 \psi^*}{\partial x^2}\right)\right] dx\\
                \\
                &= \frac{i\hbar}{2m} \int_{-\infty}^{\infty} x \left[\psi^* \frac{\partial^2 \psi}{\partial x^2} - \psi \frac{\partial^2 \psi^*}{\partial x^2}\right] dx\\
                \\
                &= \frac{i\hbar}{2m} \int_{-\infty}^{\infty} x \frac{\partial}{\partial x} \left(\psi^* \frac{\partial \psi}{\partial x} - \psi \frac{\partial \psi^*}{\partial x}\right) dx\\
            \end{split}\]
            Next, we're going to utilize integration by parts. If we let $u = x$:
            \[\begin{split}
                u &= x\\
                du &= \frac{\partial x}{\partial x}\\
                du &= 1
            \end{split}\]
            and $dv = \frac{\partial}{\partial x} \left(\psi^* \frac{\partial \psi}{\partial x} - \psi \frac{\partial \psi^*}{\partial x}\right)$:
            \[\begin{split}
                dv &= \frac{\partial}{\partial x} \left(\psi^* \frac{\partial \psi}{\partial x} - \psi \frac{\partial \psi^*}{\partial x}\right)\\
                \\
                v &= \int_{-\infty}^{\infty} \frac{\partial}{\partial x} \left(\psi^* \frac{\partial \psi}{\partial x} - \psi \frac{\partial \psi^*}{\partial x}\right) dx\\
                \\
                v &= \psi^* \frac{\partial \psi}{\partial x} - \psi \frac{\partial \psi^*}{\partial x}
            \end{split}\]
            Plugging these expressions back to our original integral:
            \[\begin{split}
                \frac{d\braket{x}}{dt} &= \frac{i\hbar}{2m} \left[\left. x \left( \psi^* \frac{\partial \psi}{\partial x} - \psi \frac{\partial \psi^*}{\partial x} \right) \right\rvert_{-\infty}^{\infty} - \int_{-\infty}^{\infty} \left( \psi^* \frac{\partial \psi}{\partial x} - \psi \frac{\partial \psi^*}{\partial x} \right)dx \right]
            \end{split}\]
            Because of normalization condition, $\left. \left( \psi^* \frac{\partial \psi}{\partial x} - \psi \frac{\partial \psi^*}{\partial x} \right) \right\rvert_{-\infty}^{\infty}$ tends to zero, thus:
            \[\begin{split}
                \frac{d\braket{x}}{dt} &= -\frac{i\hbar}{2m} \int_{-\infty}^{\infty} \left( \psi^* \frac{\partial \psi}{\partial x} - \psi \frac{\partial \psi^*}{\partial x} \right)dx
            \end{split}\]
            We are going to utilize integration by parts one more time, but only on the right half of the integral:
            \[\begin{split}
                \int_{-\infty}^{\infty} \psi \frac{\partial \psi^*}{\partial x} dx &= \left. \psi (\psi^*) \right\rvert_{-\infty}^{\infty} - \int_{-\infty}^{\infty} \psi^* \frac{\partial \psi}{\partial x} dx
            \end{split}\]
            Again, because of the normalization condition, both $\psi$ and $\psi^*$ tends to zero as $x$ approaches $\pm \infty$. Therefore we are left with:
            \[\begin{split}
                \int_{-\infty}^{\infty} \psi \frac{\partial \psi^*}{\partial x} dx &= - \int_{-\infty}^{\infty} \psi^* \frac{\partial \psi}{\partial x} dx
            \end{split}\]
            Plugging it to the original integral:
            \[\begin{split}
                \frac{d\braket{x}}{dt} &= -\frac{i\hbar}{2m} \int_{-\infty}^{\infty} \left[ \psi^* \frac{\partial \psi}{\partial x} - \left(- \psi^* \frac{\partial \psi}{\partial x}\right) \right]dx\\
                \\
                &= -\frac{i\hbar}{2m} \int_{-\infty}^{\infty} 2 \cdot \psi^* \frac{\partial \psi}{\partial x} dx\\
            \end{split}\]
            The $2$s cancels each other, and we are left with
            \begin{equation}
                \frac{d\braket{x}}{dt} = -\frac{i\hbar}{m} \int_{-\infty}^{\infty} \psi^* \frac{\partial \psi}{\partial x} dx
            \end{equation}
            Now, to move forward, we know the postulate that the derivative of the expectation value $\braket{x}$ with respect to time is just the expectation value of the velocity operator $v_x$:
            \[\braket{v_x} = \frac{d\braket{x}}{dt}\]
            And the expectation value of the momentum operator $\braket{p_x}$ can be related to $\braket{v_x}$ by:
            \[\braket{p_x} = m \braket{v_x}\]
            Substituting $\braket{v_x}$ with the derivative in $(3)$:
            \[\begin{split}
                \braket{p_x} &= m \left(-\frac{i\hbar}{m} \int_{-\infty}^{\infty} \psi^* \frac{\partial \psi}{\partial x} dx \right)\\
                \\
                &= -i\hbar \int_{-\infty}^{\infty} \psi^* \frac{\partial \psi}{\partial x} dx\\
            \end{split}\]
            Changing the constants in front of the integral into its fraction form yields:
            \begin{equation}
                \braket{p_x} = \frac{\hbar}{i} \int_{-\infty}^{\infty} \psi^* \frac{\partial \psi}{\partial x} dx
            \end{equation}

        \subsection{Momentum Operator}
            Similar to finding the position operator, we can start by first representing $\braket{p_x}$ in terms of the operator $\pxhat$:
            \begin{equation}
                \braket{p_x} = \frac{\braket{\psi | \pxhat | \psi}}{\braket{\psi | \psi}}
            \end{equation}
            Then equate $(4)$ to $(5)$ since they both are equivalent to $\braket{p_x}$:
            \[\frac{\braket{\psi | \pxhat | \psi}}{\braket{\psi | \psi}} = \frac{\hbar}{i} \int_{-\infty}^{\infty} \psi^* \frac{\partial \psi}{\partial x} dx\]
            The denominator of the left hand side is just $1$, and rearranging the right-hand side yields:
            \[\begin{split}
                \braket{\psi | \pxhat | \psi} &= \int_{-\infty}^{\infty} \psi^* \left(\frac{\hbar}{i} \frac{\partial}{\partial x}\right)\psi dx
            \end{split}\]
            From which we can finally deduce the momentum operator $\pxhat$ as:
            \[\pxhat = \frac{\hbar}{i} \frac{\partial}{\partial x}\]

    \section{Importance of Position and Momentum Operators}
        Almost all Classical Mechanics quantities can be expressed in terms of position and momentum.\n
        For instance the Kinetic Energy:
        \[E_k = \frac{1}{2}mv^2 = \frac{p^2}{2m}\]
        and others like the Potential Energy, Angular Momentum, etc.\n
        In general we can find the expectation value of almost any Classical Mechanics quantity $Q$ by using the operators $\xhat$ and $\pxhat$:
        \[\braket{Q(x,p)} = \int_{-\infty}^{\infty} \psi^* Q(\xhat, \pxhat) \psi dx\]
    \bibliographystyle{apacite}
    \bibliography{References}

\end{document}