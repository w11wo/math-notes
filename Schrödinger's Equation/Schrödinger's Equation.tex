\documentclass[hidelinks, a4paper, 12pt]{article}
\usepackage[linktoc=all]{hyperref}
\usepackage{apacite}
\usepackage[margin=1.0in]{geometry}
\usepackage{amssymb}
\usepackage{amsmath}
\usepackage{amsthm}
\newtheorem{theorem}{Theorem}
\usepackage{pgfplots}
\pgfplotsset{compat=1.16}
\usepackage{tikz}
\usepackage{pgf}
\usepackage{mathrsfs}
\usepackage{array}
\usepackage{tabularx}
\usepackage{braket}

\allowdisplaybreaks


\hypersetup{
    pdftitle={Schrödinger's Equation},
    pdfauthor={Wilson Wongso},
    pdfpagemode=UseOutlines,
}

\title{Schrödinger's Equation}
\author{Notes taken by Wilson Wongso}
\date{}

\setcounter{section}{-1}
\setcounter{tocdepth}{2}

\graphicspath{ {./images/} }

\newcommand{\biimp}{\Leftrightarrow}
\newcommand{\bd}{\textbf}
\newcommand{\n}{\\[\baselineskip]}
\newcommand{\real}{\mathbb{R}}
\newcommand{\thus}{\Rightarrow}

\begin{document}

    \maketitle
        
    \tableofcontents

    \section{Preface}
        The following notes are based on the lecture video \bd{Introduction to Quantum Mechanics: Schrödinger Equation} \cite{QM-Schrodinger}.
        The author simply wishes to compile a part of his learning journey into this document.

    \section{Correlation with Classical Mechanics}
        Suppose $p$ is a particle moving in the $x$-axis, with force $\vec{F_i}$, where $\vec{F_i}$ is a function of position and time, $\vec{F_i}(x,t)$.\n
        According to Newton's Second Law:
        \[\vec{F}_{net} = \sum_{i=1}^{n}F_i(x,t) = ma \biimp m\frac{d^2x}{dt^2}\]
        and solving the equation of motion allows us to find out the particle's velocity, kinetic energy, and other thing's about the particle's state.\n
        Solving Schrödinger's Equation also gives us similar things but that of a wavefunction instead.

    \section{Schrödinger's Equation (in 1-Dimension)}
        \[i\hbar \frac{\partial\psi}{\partial t} = \frac{-\hbar^2}{2m}\frac{\partial^2\psi}{\partial t^2}+V\psi\]
        where $\psi$ is the wavefunction, $\frac{-\hbar^2}{2m} \frac{\partial^2}{\partial t^2}$ is the kinetic energy operator and $V$ is the potential energy operator.

    \section{Wavefunction}
        Wavefunction $\psi$ represent the state of the system. $|\psi|^2$ represents the probability density function of the system.
        \subsection{Normalization Condition}
            To be normalized, the square of the absolute value of the wavefunction $|\psi|^2$ should have an area of $1$, that is:
            \[\int_{-\infty}^{\infty}|\psi|^2dx = 1\]
            which is called the \bd{normalization condition}.

        \subsubsection{Probability Density Distribution}
            \begin{tikzpicture}
                \begin{axis}[
                    axis lines=middle,
                    xlabel=$x$, xlabel style={at=(current axis.right of origin), anchor=west},
                    ylabel=$y$, ylabel style={at=(current axis.above origin), anchor=south},
                    xtick=\empty, ytick=\empty,
                    clip=false,
                ]
                \addplot [
                    domain=0.5:3.1, 
                    samples=100, 
                    color=red,
                ]
                {sin(deg(x))};
                \addplot +[mark=none, color=blue, smooth] coordinates {(1.2, 0) (1.2, 0.93)};
                \addplot +[mark=none, color=blue, smooth] coordinates {(1.9, 0) (1.9, 0.9463)};
                \node[label={150:{$Likely$}},circle,fill,inner sep=2pt] at (axis cs:1.6,1) {};
                \node[label={150:{$Unlikely$}},circle,fill,inner sep=2pt] at (axis cs:3.1,0.0416) {};
                \node[label={150:{$a$}},circle,fill,inner sep=2pt] at (axis cs:1.2, 0) {};
                \node[label={20:{$b$}},circle,fill,inner sep=2pt] at (axis cs:1.9, 0) {};
                \end{axis}
            \end{tikzpicture}\n
            The graph signifies that it is more likely (probability-wise) to find the particle at the peak of the distribution, 
            and more unlikely to find the particle at the dip of the distribution.\n
            Additionally, the probability of finding the particle at state $\psi$ between $a$ and $b$ is given by the integral:
            \[\int_{a}^{b}|\psi|^2dx\]

        \subsection{Wavefunction Collapse}
            Successive measurements keep giving $Q$, as measurements change particle wavefunction into delta function.\n
            \begin{tikzpicture}
                \begin{axis}[
                    axis lines=middle,
                    xlabel=$x$, xlabel style={at=(current axis.right of origin), anchor=west},
                    ylabel=$y$, ylabel style={at=(current axis.above origin), anchor=south},
                    xtick=\empty, ytick=\empty,
                    clip=false,
                ]
                \addplot [
                    domain=0.5:3.1, 
                    samples=100, 
                    color=white,
                ]
                {sin(deg(x))};
                \addplot [
                    domain=1.58:1.6, 
                    samples=100, 
                    color=blue,
                ]
                {50*x-79};
                \addplot [
                    domain=1.6:1.62, 
                    samples=100, 
                    color=blue,
                ]
                {-50*x+81};
                \node[label={-90:{$Q$}},circle,fill,inner sep=2pt] at (axis cs:1.6, 0) {};
                \end{axis}
            \end{tikzpicture}\n
            Letting the system settle for a long time restores it into its original wavefunction.

    \section{Statistics in Quantum Mechanics}
        If $p(x)$ is a probability density function (e.g. $|\psi|^2$), then:
        \[\begin{split}
            \int_{-\infty}^{\infty}p(x)dx &= \int_{-\infty}^{\infty}|\psi|^2dx = 1\\
            \\
            \braket{x^n} &= \int_{-\infty}^{\infty}x^n p(x)dx = \int_{-\infty}^{\infty}x^n |\psi|^2dx\\
            \\
            \sigma_x^2 &= \braket{x^2} - \braket{x}^2
        \end{split}\]

    \section{Auxiliary Condition of Schrödinger's Equation}
        Like any other Partial Differential Equations, there has to be an auxiliary/initial condition, 
        such that it satisfies the normalization constraint:
        \[\int_{-\infty}^{\infty}|\psi|^2dx = 1\]
        Solutions like $\psi(x, t) = 0$ is un-physical as the auxiliary condition isn't satisfied.\n
        Suppose we solve Schrödinger's Equation and get:
        \[\psi(x, t) = Af(x,t)\]
        where $A$ is an arbitrary constant.\n
        Normalization means finding $A$ so that:
        \[\int_{-\infty}^{\infty}|\psi|^2dx = \int_{-\infty}^{\infty}|Af|^2dx = 1\]
        and the constant $A$ needs to stay the same at different times, in order for the normalization of $\psi$ to be preserved with time.\n
        The proof of the theorem regarding the preservation of normalization is to be shown in the next section.

    \section{Preservation of Normalization of Wavefunction}
        \begin{theorem}
            Preservation of Normalization of Wavefunction
            \[\frac{d}{dt} \int_{-\infty}^{\infty}|\psi|^2dx = 0\]
        \end{theorem}
        \begin{proof}
            Recall that the norm squared of a wavefunction is given by:
            \[|\psi|^2 = (\psi^*) \psi\]
            where $\psi^*$ is its complex conjugate.\n
            Moreover, if we go back to Schrödinger's Equation:
            \begin{equation}
                i\hbar \frac{\partial\psi}{\partial t} = \frac{-\hbar^2}{2m}\frac{\partial^2\psi}{\partial t^2}+V\psi
            \end{equation}
            Then take its complex conjugate:
            \begin{equation}
                -i\hbar \frac{\partial\psi^*}{\partial t} = \frac{-\hbar^2}{2m}\frac{\partial^2\psi^*}{\partial t^2}+V\psi^*
            \end{equation}
            Rearranging $(1)$ to set $\frac{\partial\psi}{\partial t}$ as the subject:
            \begin{equation}
                \frac{\partial\psi}{\partial t} = \frac{i\hbar}{2m}\frac{\partial^2\psi}{\partial x^2} - \frac{i}{\hbar} V\psi
            \end{equation}
            Similarly, rearranging $(2)$ to set $\frac{\partial\psi^*}{\partial t}$ as the subject:
            \begin{equation}
                \frac{\partial\psi^*}{\partial t} = \frac{-i\hbar}{2m}\frac{\partial^2\psi}{\partial x^2} + \frac{i}{\hbar} V\psi^*
            \end{equation}
            Then differentiate the normalization of wavefunction with respect to time $t$:
            \[\begin{split}
                \frac{d}{dt} \int_{-\infty}^{\infty}|\psi|^2dx &= \frac{d}{dt} \int_{-\infty}^{\infty}\psi^*\psi dx\\
                &= \int_{-\infty}^{\infty}\frac{\partial}{\partial t} (\psi^*\psi) dx\\
            \end{split}\]
            Using the product rule:
            \[\begin{split}
                \int_{-\infty}^{\infty}\frac{\partial}{\partial t} (\psi^*\psi) dx &= \int_{-\infty}^{\infty} \psi^* \frac{\partial\psi}{\partial t} dx + \int_{-\infty}^{\infty} \psi \frac{\partial\psi^*}{\partial t} dx\\
            \end{split}\]
            Substituting $(3)$ and $(4)$:
            \[\begin{split}
                \int_{-\infty}^{\infty}\frac{\partial}{\partial t} (\psi^*\psi) dx &= \int_{-\infty}^{\infty} \psi^* \left(\frac{i\hbar}{2m}\frac{\partial^2\psi}{\partial x^2} - \frac{i}{\hbar} V\psi\right) dx + \int_{-\infty}^{\infty} \psi \left(\frac{-i\hbar}{2m}\frac{\partial^2\psi^*}{\partial x^2} + \frac{i}{\hbar} V\psi^*\right) dx\\
                                                                                   &= \int_{-\infty}^{\infty} \psi^* \left(\frac{i\hbar}{2m}\frac{\partial^2\psi}{\partial x^2}\right) dx - \int_{-\infty}^{\infty} \psi \left(\frac{i\hbar}{2m}\frac{\partial^2\psi^*}{\partial x^2}\right) dx\\
            \end{split}\]
            Utilizing Integration by Parts yields:
            \[\begin{split}
                &= \left.\psi^* \left(\frac{i\hbar}{2m}\frac{\partial\psi}{\partial x}\right) \right\rvert_{-\infty}^{\infty} - \int_{-\infty}^{\infty} \frac{\partial\psi^*}{\partial x} \left(\frac{i\hbar}{2m}\frac{\partial\psi}{\partial x}\right) dx - \left.\psi \left(\frac{i\hbar}{2m}\frac{\partial\psi^*}{\partial x}\right) \right\rvert_{-\infty}^{\infty} + \int_{-\infty}^{\infty} \frac{\partial\psi}{\partial x} \left(\frac{i\hbar}{2m}\frac{\partial\psi^*}{\partial x}\right) dx\\
                &= \left.\psi^* \left(\frac{i\hbar}{2m}\frac{\partial\psi}{\partial x}\right) \right\rvert_{-\infty}^{\infty} - \left.\psi \left(\frac{i\hbar}{2m}\frac{\partial\psi^*}{\partial x}\right) \right\rvert_{-\infty}^{\infty}\\
                &= 0
            \end{split}\]
            since $\psi$ and $\psi^*$ $\rightarrow 0$ as $x \rightarrow \pm \infty$ due to the normalization condition $\int_{-\infty}^{\infty}|\psi|^2dx = 1$.
        \end{proof}

    \bibliographystyle{apacite}
    \bibliography{References}

\end{document}